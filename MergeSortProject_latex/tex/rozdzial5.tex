\newpage
\section{Wnioski} % 5

Realizacja projektu pozwoliła na praktyczne zrozumienie działania algorytmu sortowania przez scalanie (\textit{Merge Sort}), rekurencji oraz mechanizmu **szablonów (\textit{templates})** w języku \texttt{C++}. Dzięki zastosowaniu programowania uogólnionego, udało się stworzyć elastyczne rozwiązanie, które poprawnie obsługuje różne typy danych liczbowych (np. \texttt{int}, \texttt{double}) bez konieczności duplikowania kodu.

W trakcie realizacji projektu zwrócono szczególną uwagę na:
\begin{itemize}
    \item implementację algorytmu o złożoności obliczeniowej $O(n \log n)$ zgodnie z paradygmatem "dziel i zwyciężaj",
    \item weryfikację poprawności oprogramowania za pomocą profesjonalnego frameworka **Google Test**,
    \item obsługę przypadków brzegowych, takich jak puste tablice, tablice jednoelementowe czy zbiory zawierające duplikaty i liczby ujemne,
    \item separację logiki aplikacji od kodu testowego poprzez podział na dwa projekty w solucji Visual Studio.
\end{itemize}

Dodatkowo, zastosowanie narzędzia \texttt{Git} oraz platformy \texttt{GitHub} umożliwiło kontrolowanie wersji projektu, bezpieczne wprowadzanie zmian oraz symulację pracy w środowisku rozproszonym. Z kolei narzędzie \texttt{Doxygen} pozwoliło na automatyczne wygenerowanie profesjonalnej dokumentacji technicznej w formacie \LaTeX, co znacząco usprawniło proces tworzenia raportu.

Przeprowadzone testy jednostkowe (łącznie 13 scenariuszy) zakończyły się pełnym sukcesem, co potwierdza niezawodność i stabilność zaimplementowanego algorytmu. Program jest gotowy do dalszego rozwoju lub integracji z większymi systemami wymagającymi efektywnego sortowania danych.

\nocite{*}